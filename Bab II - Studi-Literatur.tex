% ==========================================
% BAB II STUDI LITERATUR
% ==========================================
\chapter{STUDI LITERATUR}
\label{chap:studi-literatur}

\section{Survei Pengenalan Wajah Berbasis Deep Learning}
Sebuah survei mengenai pengenalan wajah berbasis deep learning yang menyajikan gambaran umum tentang perkembangan di bidang ini. Studi membahas berbagai arsitektur jaringan saraf tiruan, fungsi loss, dan strategi pelatihan yang digunakan untuk meningkatkan akurasi dan ketahanan sistem pengenalan wajah. Fungsi loss mempunyai peran yang vital dalam melatih model agar bisa membedakan identitas yang berbeda. Pembahasan di studi ini mencakup evolusi dari fungsi loss tradisional hingga fungsi loss berbasis margin yang meningkatkan jarak antar kelas dan penurunan jarak intra kelas.

\section{AdaFace : Quality Adaptive Margin for Face Recognition}
Penelitian ini berfokus pada penyelesaian masalah variabilitas kualitas gambar. Konsep utama dari AdaFace adalah tidak semua gambar memberikan kontribusi belajar yang sama. Contohnya adalah gambar yang berkualitas tinggi merupakan contoh yang “mudah”, sementara gamabr yang berkualitas rendah adalah contoh yang “sulit”. Metode pelatihan konvensional memperlakukan semua gambar ini secara setara, membuat model sulit belajar dari gambar berkualitas rendah. Kualitas setiap gambar dalam data training diestimasi secara langsung. Tanpa menggunakan model yang terpisah, AdaFace menggunakan norma dari feature norm yang dihasilkan dari model itu sendiri sebagai proksi untuk kualitas gambar. Gambar yang berkualitas tinggi cenderung menghasilkan fitur dengan norma yang lebih besar. Dengan estimasi kualitas, AdaFace menyesuaikan margin pada fungsi loss. Untuk gambar berkualitas tinggi, margin yang lebih besar diterapkan agar model bekerja lebih keras dan menghasilkan fitur yang lebih diskriminatif. Sebaliknya, untuk gambar berkualitas rendah, margin yang lebih kecil dilakukan. Hal ini mencegah terpengaruh oleh noise dan artefak pada gambar tetapi tetap bisa mempelajari fitur identitas dasarnya.

\section{Android Mobile Security and File Protection Using Face Recognition}
Penelitian ini mengubah fokus dari pengembangan algoritma menjadi implementasi praktik pada platform Android. Studi ini mendemonstrasikan kelayakan dan tantangan dalam menerapkan teknologi pengenalan wajah untuk tujuan keamanan. Untuk dapat berjalan secara efisien di perangkat seluler, model CNN yang digunakan harus dioptimalkan. Arsitektur yang dirancang khusus untuk perangkat edge, seperti MobileNet atau SqueezeNet, yang menyeimbangkan akurasi dengan kecepatan komputasi dan penggunaan memori yang rendah.

\section{Efficient Face Recognition System for Operating in Unconstrained
Environments}
Penelitian ini berasal dari masalah bahwa banyak sistem pengenalan wajah yang sudah akurat, tetapi belum efisien atau tidak stabil jika dalam kondisi linkungan yang tidak terkendali. Sebaliknya, sistem yang ringan dan cepat sering kali tidak mampu mempertahankan akurasi tinggi. Berdasarkan ini, penelitian dilakukan agar sistem pengenalan wajah tetap akurat meskipun dijalankan di linkungan tidak terkendali. Deteksi wajah berbasis deep learning seperti You Only Look Once yang cepat dan efisien, ekstraksi fitur wajah berbasis embedding seperti FaceNet, yang bisa merepresentasikan wajah kedalam vektor numerik dengan jarak Euclidean. Penelitian juga menggunakan algoritma klasifikasi tradisional seperti SVM, KNN, dan Random Forest untuk menggantikan layer softmax pada jaringan FaceNet.

\section{Face Recognition Using Facial Features}
Dalam penelitian ini, masalah yang diangkat adalah metode pengenalan wajah masih bersifat “black box” dan kurang menjelaskan fitur wajah yang memengaruhi hasil pengenalan. Selain itu, sebagian metode deep learning memerlukan data dan daya komputasi yang besar, sementara metode berbasis geometri dan struktural untuk fitur wajah masih jarang dikaji dengan pendekatan modern. Penelitian ini mencoba menyelesaikan masalah untuk bagaimana menggunakan fitur waja utama untuk pengenalan yang ringan dan akurat. Penelitian mendasarkan penelitian pada teori ekstraksi fitur geometris dan landmark wajah, di mana posisi dan jarak antar fitur wajah menjadi representasi identitas seseorang. Metode ini menggunakan algoritma seperti Active Shape Model (ASM) atau Active Appearance Model (AAM) untuk mendeteksi landmark dan Local Binary Pattern (LBP) atau Histogram of Oriented Gradients (HOG) untuk mengekstrak tekstur. Konsep ini menggabungkan pendekatan analisis bentuk dan tekstur sebagai tekstur pengenalan.