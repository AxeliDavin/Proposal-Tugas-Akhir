% ============================================================================================
% BAB III ANALISIS MASALAH
% Pembagian subbab tidak rigid dan dapat bervariasi. Bab ini minimal berisi analisis kebutuhan
% fungsional dan nonfungsional, analisis berbagai alternatif solusi yang dapat ditawarkan, dan
% metode pemilihan solusi yang diusulkan.
% ============================================================================================
\chapter{ANALISIS MASALAH}
\label{chap:analisis-masalah}

\section{Analisis Kondisi Saat Ini}
Model konseptual sistem kontrol akses yang ada saat ini di Gedung ITB Innovation Park (IIP) masih mengandalkan, atau setidaknya mempertimbangkan, metode konvensional. Komponen utamanya adalah kunci fisik dan kartu akses. Ketiadaan sistem modern yang terimplementasi secara optimal ini menimbulkan beberapa masalah fundamental.

Analisis kondisi saat ini didasarkan pada observasi lapangan serta analisis dokumen yang relevan. Dari observasi, teridentifikasi bahwa titik-titik akses utama belum dilengkapi sistem keamanan terotomatisasi. Analisis dokumen mencakup tinjauan terhadap Standar Operasional Prosedur (SOP) keamanan gedung dan denah arsitektur untuk memetakan alur pergerakan manusia serta posisi strategis untuk penempatan sistem kontrol. Gambar \ref{fig:denah-iip} menunjukkan denah Gedung IIP dan titik-titik akses yang menjadi fokus dalam penelitian ini.

\begin{figure}[H] % Menggunakan [H] agar gambar tidak 'mengambang'
    \centering
    % Ganti 'placeholder-denah.png' dengan nama file gambar 
    % \includegraphics[width=0.8\textwidth]{images/placeholder-denah.png} 
    \framebox(300,150){Placeholder: Denah Gedung IIP dan Titik Akses}
    \caption{Denah Gedung IIP dan Titik Akses Kritis}
    \label{fig:denah-iip}
\end{figure}

Masalah utama dari sistem konvensional ini adalah tingkat keamanannya yang terbatas. Metode kunci fisik dan kartu akses sangat rentan terhadap risiko umum seperti kehilangan, pencurian, atau duplikasi yang tidak sah. Mengingat Gedung IIP berfungsi sebagai pusat inovasi yang menyimpan berbagai aset bernilai tinggi, celah keamanan ini menciptakan sebuah kerentanan keamanan yang nyata dan mendesak untuk segera diatasi. Sistem yang ada saat ini (atau ketiadaan sistem yang memadai) dinilai tidak lagi efisien dalam pengelolaan dan tidak mampu melindungi aset di dalamnya secara optimal.

\section{Analisis Kebutuhan}
Proses analisis kebutuhan sistem dilakukan untuk mendefinisikan apa yang harus dilakukan oleh sistem guna menyelesaikan masalah yang ada. Metode pengumpulan kebutuhan yang digunakan adalah:
\begin{enumerate}
    \item {Wawancara Semi-Terstruktur:} Dilakukan dengan pemangku kepentingan, yaitu perwakilan dari Manajemen/Pengelola Gedung IIP untuk memahami kebutuhan dari sisi keamanan dan operasional, serta wawancara dengan beberapa Penghuni Gedung (staf/peneliti) untuk memahami kebutuhan dari sisi pengguna akhir.
    \item {Observasi Langsung:} Mengamati secara langsung proses keluar-masuk gedung pada jam-jam sibuk untuk mengidentifikasi potensi masalah alur dan kebutuhan non-fungsional seperti kecepatan akses.
\end{enumerate}
Hasil dari kedua metode ini digunakan untuk merumuskan masalah pengguna serta kebutuhan fungsional dan non-fungsional.

\subsection{Identifikasi Masalah Pengguna}
Berdasarkan analisis kondisi saat ini dan metode pengumpulan kebutuhan, terdapat dua kelompok pengguna utama dengan masalah yang spesifik:
\begin{enumerate}
    \item Penghuni Gedung (Pegawai, Anggota Terdaftar, Staf)
        \begin{enumerate}
            \item Mengalami kesulitan dengan metode akses konvensional yang merepotkan (harus membawa kartu/kunci) dan tidak higienis (harus menyentuh perangkat bersama).
            \item Membutuhkan sistem yang memberikan "kemudahan bagi pengguna".
            \item Memiliki risiko keamanan pribadi jika kunci atau kartu akses mereka hilang atau diduplikasi.
        \end{enumerate}

    \item Manajemen/Pengelola Gedung IIP
        \begin{enumerate}
            \item Menghadapi masalah utama berupa kerentanan keamanan terhadap aset bernilai tinggi.
            \item Kesulitan mengelola dan melacak hak akses secara efisien menggunakan metode konvensional.
            \item Kekurangan data operasional mengenai pola keluar-masuk, yang dapat digunakan untuk efisiensi.
        \end{enumerate}
\end{enumerate}

\subsection{Kebutuhan Fungsional}
Berdasarkan rumusan masalah dan tujuan, kebutuhan fungsional (KF) untuk prototipe yang diusulkan adalah sebagai berikut:
\begin{enumerate}
    \item KF-1: Sistem harus dapat mengakuisisi citra wajah pengguna secara real-time melalui sensor kamera.
    \item KF-2: Sistem harus dapat melakukan proses otentikasi (verifikasi) dengan membandingkan citra wajah yang ditangkap dengan basis data pengguna yang terdaftar.
    \item KF-3: Sistem harus dapat memberikan perintah untuk membuka aktuator (kunci elektronik) jika otentikasi pengguna berhasil (memberikan hak akses).
    \item KF-4: Sistem harus dapat menolak akses (tidak mengirim perintah ke kunci elektronik) jika otentikasi gagal atau wajah tidak terdaftar.
\end{enumerate}

\subsection{Kebutuhan Nonfungsional}
Kebutuhan nonfungsional (KNF) didefinisikan secara spesifik dalam kriteria keberhasilan tugas akhir ini:
\begin{enumerate}
    \item KNF-1 (Kecepatan): Waktu yang dibutuhkan sistem untuk menyelesaikan satu siklus proses otentikasi—mulai dari deteksi wajah hingga pengiriman perintah ke kunci elektronik—harus kurang dari 3 detik.
    \item KNF-2 (Akurasi): Sistem harus mampu melakukan otentikasi wajah pengguna yang terdaftar dengan tingkat akurasi di atas 95\% pada kondisi pengujian yang terkontrol.
    \item KNF-3 (Keandalan): Sistem harus mampu secara konsisten membedakan antara pengguna terdaftar (memberikan akses) dan pengguna tidak terdaftar (menolak akses).
\end{enumerate}

\section{Analisis Pemilihan Solusi}

\subsection{Alternatif Solusi}
Berdasarkan penelusuran literatur dan studi kasus pada sistem kontrol akses modern, diidentifikasi tiga alternatif solusi utama yang relevan untuk dibandingkan:

\subsubsection{Solusi Konvensional (Kartu Akses/RFID)}
\begin{enumerate}
    \item {Konsep Dasar:} Menggunakan kartu berbasis *Radio-Frequency Identification* (RFID) yang di-tap pada sebuah pembaca (reader) untuk memverifikasi hak akses.
    \item {Kelebihan:} Biaya implementasi per unit relatif rendah, teknologi matang, dan kecepatan otentikasi sangat cepat (kurang dari 1 detik).
    \item {Keterbatasan:} Keamanan terbatas (kartu dapat hilang, dicuri, atau diduplikasi), merepotkan pengguna (harus selalu membawa kartu), dan biaya operasional untuk penggantian kartu yang hilang.
\end{enumerate}

\subsubsection{Solusi Biometrik (Sidik Jari)}
\begin{enumerate}
    \item {Konsep Dasar:} Menggunakan pemindai untuk merekam pola unik sidik jari pengguna dan mencocokkannya dengan basis data.
    \item {Kelebihan:} Tingkat keamanan tinggi (sulit dipalsukan), tidak memerlukan perangkat tambahan yang harus dibawa pengguna.
    \item {Keterbatasan:} Memerlukan kontak fisik (tidak higienis, menjadi masalah terutama pasca-pandemi), rentan gagal baca jika jari kotor atau basah, dan dapat menimbulkan antrian karena proses *scan* yang individual.
\end{enumerate}

\subsubsection{Solusi Biometrik (Pengenalan Wajah)}
\begin{enumerate}
    \item {Konsep Dasar:} Menggunakan kamera untuk menangkap fitur wajah pengguna, kemudian algoritma *computer vision* memverifikasi identitas pengguna tersebut.
    \item {Kelebihan:} Bersifat *non-kontak* (higienis dan nyaman), keamanan tinggi, dan tidak memerlukan perangkat tambahan. Proses bisa berjalan pasif tanpa interaksi eksplisit dari pengguna.
    \item {Keterbatasan:} Biaya implementasi awal bisa lebih tinggi (memerlukan kamera dan unit pemrosesan yang mumpuni), akurasi dapat dipengaruhi oleh kondisi pencahayaan ekstrem, penggunaan masker, atau perubahan penampilan drastis.
\end{enumerate}

\subsection{Analisis Penentuan Solusi}
Untuk memilih solusi terbaik secara sistematis dan objektif, digunakan metode \textit{Weighted Scoring Model} (WSM). Kriteria penilaian dan bobotnya ditentukan berdasarkan prioritas kebutuhan yang diidentifikasi pada bagian Analisis Kebutuhan. Kriteria utama meliputi {Keamanan} (Bobot: 40\%), {Kemudahan Pengguna/Higienitas} (Bobot: 30\%), {Kecepatan Akses} (Bobot: 20\%), dan {Biaya Implementasi Awal} (Bobot: 10\%).

Setiap alternatif solusi diberi skor 1 (Sangat Buruk) hingga 5 (Sangat Baik) untuk setiap kriteria.

\begin{table}[H]
\centering
\caption{Analisis Penentuan Solusi Menggunakan Weighted Scoring Model}
\label{tab:wsm-solusi}

\small
\setlength{\tabcolsep}{4pt}

\begin{tabular}{l c c c c}
\toprule
\textbf{Kriteria Penilaian} & \textbf{Bobot} & \textbf{Kartu Akses (RFID)} & \textbf{Sidik Jari} & \textbf{Pengenalan Wajah} \\
\midrule
Keamanan & 40\% & Skor: 2 (1.0) & Skor: 4 (1.6) & Skor: 4 (1.6) \\
Kemudahan/Higienitas & 30\% & Skor: 3 (0.9) & Skor: 2 (0.6) & Skor: 5 (1.5) \\
Kecepatan Akses & 20\% & Skor: 5 (1.0) & Skor: 3 (0.6) & Skor: 4 (0.8) \\
Biaya Implementasi & 10\% & Skor: 5 (0.5) & Skor: 4 (0.4) & Skor: 3 (0.3) \\
\midrule
\textbf{Total Skor} & \textbf{100\%} & \textbf{3.4} & \textbf{3.2} & \textbf{4.2} \\
\bottomrule
\end{tabular}
\end{table}

Berdasarkan hasil pada Tabel \ref{tab:wsm-solusi}, {Solusi Pengenalan Wajah} memperoleh total skor tertinggi (4.2). Meskipun sedikit lebih mahal dalam implementasi awal (Skor: 3), keunggulannya yang signifikan pada kriteria {Kemudahan Pengguna/Higienitas} (Skor: 5) dan {Keamanan} (Skor: 4) menjadikannya pilihan yang paling sesuai untuk diterapkan di lingkungan modern seperti Gedung IIP, sejalan dengan penyelesaian masalah utama yang telah diidentifikasi.